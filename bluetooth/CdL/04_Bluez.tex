\section{BLE dans Linux}

\begin{frame}
	\frametitle{BlueZ}
\end{frame}

\begin{frame}
	\frametitle{BLE dans BlueZ}
\end{frame}

\begin{frame}
	\frametitle{GATT dans BlueZ}
\end{frame}

\begin{frame}
	\frametitle{Outils pratiques}
	\begin{minipage}{0.46\linewidth}
		\begin{block}{bluetoothctl}
			\begin{itemize}
				\item UI de bluetoothd
				\item Gestion des appareils
				\item Gestion des profils
			\end{itemize}
		\end{block}
		\begin{block}{btmon}
			\begin{itemize}
				\item Monitore HCI
				\item Monitore MGMT
				\item Excellent pour le debug
			\end{itemize}
		\end{block}
	\end{minipage}
	\begin{minipage}{0.46\linewidth}
		\begin{block}{btmgmt}
			\begin{itemize}
				\item Utilise la MGMT API
				\item Gestion du controller
				\item Gestion du dual-mode
			\end{itemize}
		\end{block}
		\begin{block}{GATT}
			\begin{itemize}
				\item gatttool
				\item btgatt-client
				\item btgatt-server
			\end{itemize}
		\end{block}
	\end{minipage}
	\vspace{0.5cm}
	\small{

	\textbf{A voir aussi} : \textit{obexctl}, \textit{rfcomm}, \textit{l2ping}, \textit{hciattach}

	\textbf{Déprécié} : \textit{hciconfig}, \textit{hcitool}, \textit{hcidump}, \textit{sdptool}
	}

\end{frame}
