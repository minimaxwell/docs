\documentclass{smilebeamer}

\usepackage[english]{babel}
\usepackage[utf8]{inputenc}
\usepackage{eurosym}
\usepackage{hyperref}
\usepackage{minted}

\title{Measuring the impacts of the Preempt-RT patch}
\author{maxime.chevallier@smile.fr}

\begin{document}

% Page titre, presentation de la société et de moi
\begin{frame}[plain]
	\title{Measuring the impacts of the Preempt-RT patch}
    \titlepage
\end{frame}

\begin{frame}
	\frametitle{RT Linux projects}
		\begin{itemize}
			\item \textbf{Simulation platform} : bi-xeon, lots ot RAM \\ \small{\texttt{200µs wakeup latency, networking}}
			\vspace{0.3cm}
			\item \textbf{Test bench} : Intel atom \\ \small{\texttt{1s max latency, I/O and networking}}
			\vspace{0.3cm}
			\item \textbf{Embedded telematic board} : i.mx6q \\ \small{\texttt{Never lose incoming data}}
			\vspace{0.3cm}
			\item \textbf{Image processing} : Intel i3 \\ \small{\texttt{Process each frame with a deadline}}
		\end{itemize}
\end{frame}

\begin{frame}
	\frametitle{What is a RTOS ?}

	\textbf{Real Time : Determinism}

	\begin{itemize}
		\item \textbf{Bounded Latencies} \\ \small{\textit{We need guaranties on the reaction time}}
			\vspace{0.3cm}
		\item \textbf{RT Scheduler} \\ \small{\textit{We want absolute priorities for the tasks}}
			\vspace{0.3cm}
		\item \textbf{Handle the complex cases} \\ \small{\textit{Priority Inversion, Starvations, etc.}}
	\end{itemize}

	{\small{Understanding a Real-Time System - Steven Rostedt}}

\end{frame}

\begin{frame}
	\frametitle{Linux}

	\textbf{We have :}

	\begin{itemize}
		\item \textbf{RT Scheduler} {\small{\texttt{SCHED\_FIFO, SCHED\_RR, SCHED\_DEADLINE}}}
		\item \textbf{PI mutexes} {\small{\texttt{futex, rt-mutex}}}
		\item \textbf{Preemptible kernel} (almost)
		\item \textbf{High resolution timers}  {\small{\texttt{nanosleep}}}
	\end{itemize}

	\textbf{We lack :}

	\begin{itemize}
		\item Full kernel preemption
		\item Some worst case scenario optimisations
	\end{itemize}
\end{frame}

% Use case réels ?

% Presenter de quoi on va parler :
% Lorsque l'on conçoit un systeme embarqué, differents choix découlent des specs :
% Hardware : CPU, mémoire, périphériques selon besoins de ressources
% Software : Noyau, librairies, application, build-system selon l'existant, temps de dev, maintenance, stabilité
% On peut choisir d'utiliser linux pour beaucoup de raisons. Si en plus de ces raisons, on a besoin de déterminisme
% et de comportement RT, alors on considere preempt-RT. mais linux-RT != linux. quelles differences? Comment les mesurer,
% Ou chercher ? En plus de devoir maitriser son systeme pour développer son appli RT avec toutes les contraintes qui vont
% avec, que reste-t-il pour le reste de mes specs ?

%\begin{frame}
%	\frametitle{Agenda}
%	\begin{itemize}
%		\item Pre-requisites
%		\item What Preempt-RT changes
%		\item How to measure that
%		\item System configuration
%	\end{itemize}
%\end{frame}

% Presenter preempt-RT
% qui qui a fait, comment ça s'applique sur le noyau, status actuel
% liste de ressources relatives a preempt-RT, comment le benchmarquer pour être
% certain qu'on tient les specs, comment ça marche : plein de ressources disponibles.

% idees reçues : Impact sur les perfs, instabilité, pas maintenu, difficile de dev kernel.


%Include each section
\begin{frame}
\tableofcontents
\end{frame}

\begin{frame}
	\frametitle{What about non-RT tasks ?}
	\begin{itemize}
		\item The kernel internals are changed
		\item Kernel-userspace API/ABI stays the same
		\item We have what is left of the resources :
			\begin{itemize}
				\item SCHED\_OTHER runs when no RT tasks run, whatever their priority
				\item User configuration might dedicate some resources to RT tasks
			\end{itemize}
	\end{itemize}
\end{frame}

\begin{frame}
	\frametitle{Firt steps}
	\begin{itemize}
		\item \textbf{Am I really running the RT patch ?} uname -a \\ \small{\texttt{cat /sys/kernel/realtime}}
		\item \textbf{More tasks are running} htop \\ \small{\texttt{Threaded IRQs - beware of load-avg}}
	\end{itemize}
\end{frame}

\begin{frame}
	\frametitle{perf}
	Performance analysis tool for Linux (from manpage)
	\begin{itemize}
		\item Uses the kernel performance counters
		\item Generate traces
		\item Versatile tool :
			\begin{itemize}
				\item debugging
				\item profiling
				\item benchmarking
			\end{itemize}
	\end{itemize}
\end{frame}

\begin{frame}
	\frametitle{perf - Vanilla linux}
	\textbf{ping -f $<$ip$>$ -c 1000000}

\texttt{3.26\%  ping  \_raw\_spin\_lock\_irqsave}

\texttt{2.40\%  ping  entry\_SYSCALL\_64}

\texttt{2.33\%  ping  \_raw\_spin\_lock}

\texttt{2.26\%  ping  fib\_table\_lookup}

\texttt{1.87\%  ping  insert\_work}

\texttt{1.62\%  ping  \_raw\_spin\_unlock\_irqrestore}

\texttt{1.60\%  ping  \_\_ip\_route\_output\_key\_hash}

\texttt{1.56\%  ping  \_\_netif\_receive\_skb\_core}

\texttt{1.53\%  ping  queue\_work\_on}
\end{frame}


\begin{frame}
	\frametitle{perf - RT Linux}
	\textbf{ping -f $<$ip$>$ -c 1000000}

\texttt{5.53\%  ping check\_preemption\_disabled}

\texttt{4.29\%  ping migrate\_enable}

\texttt{3.29\%  ping \_\_bitmap\_equal}

\texttt{2.56\%  ping migrate\_disable}

\texttt{2.55\%  ping rt\_spin\_lock}

\texttt{2.30\%  ping preempt\_count\_add}

\texttt{2.29\%  ping rt\_spin\_unlock}

\texttt{1.81\%  ping entry\_SYSCALL\_64}

\texttt{1.28\%  ping preempt\_count\_sub}
\end{frame}


\begin{frame}
	\frametitle{pidstat, vmstat, mpstat}
Event analysis tools
	\begin{itemize}
		\item Analyse context switching
		\item Interruptions
		\item Cache misses
		\item Page faults
		\item branch prediction
	\end{itemize}
\end{frame}

\begin{frame}
	\frametitle{*stat}
	\begin{columns}
		\begin{column}[t]{0.5\linewidth}
\textbf{vmstat 1}

\texttt{r in   cs}

\texttt{1 2841 696381 }

\texttt{2 2134 686653 }

\texttt{2 1511 740010 }

			\vspace{1cm}
\textbf{pidstat -w 1}

\texttt{cswch/s nvcswch/s  Command }

\texttt{70443   76 stress-ng-fifo }

\texttt{70571   61 stress-ng-fifo }

\texttt{70587   52 stress-ng-fifo }

		\end{column}
		\begin{column}[t]{0.5\linewidth}
			\begin{itemize}
				\item \textbf{vmstat} \\ {\small{\texttt{Global memory stats}}}
				\item \textbf{mpstat} \\ {\small{\texttt{per processor stats}}}
				\item \textbf{pidstat}\\ {\small{\texttt{per task stats}}}
			\end{itemize}
		\end{column}
	\end{columns}
\end{frame}

\begin{frame}
	\frametitle{Performance impacts : Preempt-RT}
	\begin{itemize}
		\item Syscalls : Expect an overhead
		\item Locks : Futexes are made faster
		\item Fifos, mqueues, pipes : Tend to get slower
	\end{itemize}
\end{frame}

\begin{frame}
	\frametitle{Performance impacts : Platform-dependent tweaking}
\begin{itemize}
	\item \textbf{CPU Idle states} : Use Poll or C1 \\ \small{\texttt{Increase power consumption}}
	\vspace{0.3cm}
	\item \textbf{Dynamic Voltage and Frequency Scaling} : Use a fixed frequency \\ \small{\texttt{Might increase power consumption}}
	\vspace{0.3cm}
	\item \textbf{Hyperthreading} : Disable it \\ \small{\texttt{Less processing power}}
\end{itemize}
	Reminder : Know your system (DMA, Resource sharing, SMM)	
\end{frame}

\begin{frame}
	\frametitle{cpuidle, cpufreq}
	\textbf{cpuidle in sysfs} : \texttt{/sys/devices/system/cpu/cpuX/stateY/}
	\begin{itemize}
		\item \texttt{name}
		\item \texttt{latency} : \textit{wakeup latency}
		\item \texttt{residency} : \textit{sleep time needed to enter}
		\item \texttt{power} : \textit{power consumed in that state}
	\end{itemize}

	\vspace{1cm}
	\textbf{powertop}

	Allows to see C-state and frequency usage
\end{frame}

\begin{frame}
	\frametitle{networking}

	\textbf{lmbench} : Micro benchmarks suite

	\begin{itemize}
		\item \textbf{tcp\_lat, udp\_lat} : Hot potato benchmark
		\item Easy to combine with perf
	\end{itemize}

	\textbf{iperf}

\end{frame}


%Final frame
\begin{frame}
	\frametitle{C'est fini}
	\center{\huge{Merci}}
\end{frame}

\end{document}

